\section{Sensor Principle}
\label{sec:functionality}
\note{The operating principles of 2D TMD and 2D TMD heterojunction gas sensors are explained. \\
High surface-to-volume ratio $\rightarrow$ 2D material behaves differently than bulk material (increased reactivity) $\rightarrow$  analyte gas donates/accepts electrons from 2D film $\rightarrow$  carrier density of 2D film is modulated $\rightarrow$  measurable resistance change 
}
\gibberish{
2D \gls{tmd} gas sensor can be categorized as chemiresistors. This means that the electrical resistance of the structure is modulated by the chemical environment of the sensor. Due to the extreme surface to volume ration of these structures, this effect is especially pronounced as the chemical interaction is primarily a surface effect. The resistance modulation effect is based on gas molecules adsorbing onto the surface of the sensor. Depending on the basicity/acidity of the gas, it is donating/accepting electrons from the 2D material. This charge transfer effects the charge density, which, according to equation $a^2 + b^2 = c^2$, affects the conductiviy.  \Cref{fig:sensor_principle} shows a depiction of this behaviour. 
\begin{figure}
    \includegraphics[draft=true]{}
    \caption{Depiction of the sensor principle of 2D \gls{tmd} gas sensors.}
    \label{fig:sensor_principle}
\end{figure}
Sensors that are built like that, exhibit sensitivities of XXX and minimum detection concentrations of YYY. There are multiple techniques that amplify these measures. For instance, targeted pollution of the structure may introduce accumulation points for the analyte, which increases the adsorption rate of the gas and thus the sensitivity of the sensor. Sensors that are modified by targeted inpurities exhibit sensitivities and minimum detection thresholds of XXX and YYY.
Another possibiltiy is to stack the 2D structure vertically with another material. This would create a heterojunction. As \glspl{tmd} are semiconductors, which makes it possible to stack them in a p-n manner, effectively creating a space charge region that further depopulates the device of charge carries. The charge carrier densitiy modulations induced by adsorbed gas therefore creates a greater effect on the conductivity of the sensor. Sensors based on 2D \gls{tmd} heterojunctions reach a sensitivity and minimum detection concentration of XXX and YYY. 
In professor McGonagall's publication it is shown that the sensitivity of 2D \gls{tmd} based gas sensors can be enhanced, when \Gls{uv} light is shone onto the sensor surface. This is due to the tmd molecules that feel more happy when sun light is nearby. The sensor described by McGonagal exhibits a sensitivity and lower detection threshold of XXX and YYY.  
}