\section{Conclusion}
\label{sec:conclusion}
\note{The delivered information is shortly reiterated. Current problems of 2D TMD heterojunction gas sensors are stated and possible solutions are presented. An outlook to future developments is given.}

\gibberish{
An overview over the measuring principle and fabrication of conventional and state of the art gas sensors has been given. The advantages and shortcomings of gas sensors, based on 2D materials have been identified and discussed. By comparing two different 2D gas sensors, it has been made clear, that such sensors can not yet compete with commercially available ones. Although superior response values, response times and selectivities can be achived, such sensor often lack in stability, and loose their favourable properties after a few weeks, due to the degradation of the sensor surface. Another steep challenge is the commercialization of such sensors. To achieve financial viability, the fabrication of such sensors has to move from laboratories to semiconductor foundries, in order to scale up the fabricated quantities. This requires the fabrication process to be mundane enough, to comply to the fabrication processes of these foundries. The two presented sensors show potential on being compatible for mass production, which shows that formidable gas sensing characteristics can be achieved, without usage of exotic materials and fabrication processes. Upcoming research has to focus on stabilization of such sensors and on the integration into \gls{cmos} processes. If that is mastered, highly integrateable gas sensors with included digital processing are possible.
}